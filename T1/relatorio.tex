\documentclass[a4paper, 11pt]{article}

\usepackage[utf8]{inputenc}

\author{Thiago da Costa Pereira Senff}
\title{Trabalho 01 - SQL e álgebra relacional }

\begin{document}
    \maketitle
    \tableofcontents

    \section{Questão}
        $ \pi_{id, populacao}(cidade) $

    \section{Questão}
        $ \sigma_{populacao<130000}(cidade) $

    \section{Questão}
        $ \pi_{id, populacao}(\sigma_{populacao<130000}(cidade)) $

    \section{Questão}
        $ cidade \bowtie_{cidade.id = cidademar.populacao} cidademar $

    \section{Questão}
        $ 
            (\pi_{pais.nome}( \sigma_{paiscontinente.continente='Europa'}(pais\bowtie_{pais.codigo=paiscontinente.pais} paiscontinente) ) )
            \cap 
            (\pi_{pais.nome}(pais \bowtie_{pais.codigo=paisreligiao.pais} paisreligiao))  
        $
        
    \section{Questão}
        $
            (\pi_{}(lingua) - ( \sigma_{pais}(paislingua)  ) )  
            (\pi_{}(lingua))  
            (\pi_{pais.nome}(pais \bowtie_{pais.codigo=paisreligiao.pais} paisreligiao))  
            (\pi_{pais.nome}(pais \bowtie_{pais.codigo=paisreligiao.pais} paisreligiao))  
        $

\end{document}



